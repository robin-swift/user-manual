\documentclass[a4paper,11pt]{article}

\usepackage[margin=1.4cm]{geometry}
\usepackage{makeidx}
\usepackage{hyperref}

\newcommand{\embversion}{2.0.0-alpha}
\newcommand{\embname}{\emph{Embroidermodder 2}}

\title{Embroidermodder 2.0 User Manual}
\author{The Embroidermodder Team}

\makeindex

\begin{document}

\maketitle

\tableofcontents

\section{To Do}

\begin{itemize}
\item Copy of license in appendix.
\item Copy of manpage in appendix.
\end{itemize}

\section{Introduction}

THIS MANUAL IS IN PROGRESS: PLEASE WAIT FOR THE BETA RELEASE.

This manual is for the various ways of using the \embname tools for designing,
editing and converting machine embroidery files. Most users should try loading and altering a design
in \embname itself before trying any of our conversion tools, our embedded system
or the command line interface. Advice on this is in the next section.

If you wish to write your own
software that uses these tools you will need the \embname Reference Manual (this
includes the API documentation). This is maintained at the permalink
\url{https://www.libembroidery.org/downloads/emrm.pdf}.

\section{Editing Designs}

WARNING: THIS FEATURE IS NOT FUNCTIONAL, THIS SECTION IS PLANNING.

\subsection{Command Line Editing}

\section{Creating a Design}

WARNING: THIS FEATURE IS NOT FUNCTIONAL, THIS SECTION IS PLANNING.

\section{Batch Conversion}

WARNING: THIS FEATURE IS NOT FUNCTIONAL, THIS SECTION IS PLANNING.

Many users of \texttt{embroider} will only want our batch conversion feature.

\section{Full Colour Photograph, Vector and Black and White Designs}

WARNING: THIS FEATURE IS NOT AT ALL PRESENT, THIS SECTION IS PLANNING.

If you are starting an embroidery from an image, it's important to note that \embname uses very different
approaches based on what kind of image is fed into it. A photograph is the most difficult starting point
for an automated system since there are many artistic decisions which are mutually exclusive and
no software should make those decisions for you. At least, you should be aware that a decision is being made.
On the other hand, a vector based design gives the program the best starting point while still not being
a machine embroidery file. Finally a scan of a vector image or hand-drawn inkwork is somewhere in-between
these options and is the recommended option for people who work best on paper.

The following subsections are in order of how good the final results should be.

\subsection{Vector Designs}
\index{vector}


\subsection{Scans of Hand-drawn Designs in a Limited Palette}
\index{scans}


\subsection{Scans of Hand-drawn Designs in Block Colours}
\index{scans}


\subsection{Full Colour Photographs}
\index{photo}



\subsection{Command Line Generation}

\section{Contributing}

\appendix

\section{GNU Free Documentation License}

\section{\texttt{embroider} UNIX Manpage}

\printindex

\end{document}
